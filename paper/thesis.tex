\RequirePackage[l2tabu, orthodox]{nag}

\documentclass[a4paper, 14pt, oneside, openany]{memoir}


\usepackage{fontspec}
\usepackage{microtype}
\usepackage{hyperref}
\usepackage{polyglossia}
\usepackage{geometry}

\geometry{a4paper, top=2cm, bottom=2cm, left=3cm, right=1.5cm, nofoot, nomarginpar}

\setmainlanguage[babelshorthands=true]{russian}
\addto\captionsrussian{\renewcommand{\contentsname}{СОДЕРЖАНИЕ}}

% В MS Office используются неправильные межстрочные интервалы. 
% В латехе межстрочный интверал соответствует определению по ГОСТ 2382-23-2004.
\OnehalfSpacing
\setmainfont{Times New Roman}

\hypersetup{
  linktocpage=true,
  colorlinks,
  linkcolor=[rgb]{0.9,0,0},
  pdflang={ru},
}

\renewcommand{\cftchapterdotsep}{\cftdotsep}
\renewcommand{\cftchapterpagefont}{\normalfont}
\renewcommand{\cftchapterleader}{\cftdotfill{\cftchapterdotsep}}
\renewcommand{\cftchaptername}{\chaptername\space}
\renewcommand{\cftchapteraftersnum}{.\space}

\newcommand{\basesectionfont}{\fontsize{14pt}{16pt}\selectfont\bfseries}

\makeatletter

\pagestyle{plain}

\makechapterstyle{thesis}{
  \chapterstyle{default}
  \setlength{\beforechapskip}{0pt}
  \setlength{\midchapskip}{0pt}
  \setlength{\afterchapskip}{\the\dimexpr 14 pt * 2}
  \renewcommand*{\chapnamefont}{\basesectionfont}
  \renewcommand*{\chapnumfont}{\basesectionfont}
  \renewcommand*{\chaptitlefont}{\basesectionfont}
  \renewcommand*{\chapterheadstart}{}

  \renewcommand*{\printchapternonum}{\center}

  \renewcommand*{\printchaptername}{\center \chapnamefont \@chapapp \space \thechapter}
  \renewcommand*{\chapternamenum}{}
  \renewcommand*{\afterchapternum}{.\space}
  \renewcommand*{\printchapternum}{}
}

\makeatother

\chapterstyle{thesis}

\setsecheadstyle{\basesectionfont\center}
\AtBeginDocument{% без этого polyglossia сама всё переопределяет
  \setsecnumformat{\csname the#1\endcsname\quad}
}

\begin{document}

\thispagestyle{empty}
\begin{center}
  Федеральное государственное автономное образовательное учреждение \\
  высшего образования \\
  <<Московский физико-технический институт (национальный исследовательский университет)>>

  \smallskip
  {\small
    Физтех-школа прикладной математики и информатики \\
    Кафедра анализа данных
  }
\end{center}

\bigskip
{
  \small\noindent
  \textbf{Направление подготовки:} 09.04.01 Информатика и вычислительная техника \\
  \textbf{Направленность (профиль) подготовки:} Анализ данных и разработка информационных систем
}

\vspace{0pt plus 8fill}

\begin{center}
  \textbf{
    \MakeUppercase{Построение эффективной распределенной очереди}\\
  }
  (магистерская диссертация)

  \vspace{0pt plus 3fill}

  \begin{flushright}
    \begin{tabular}{l}
      \textbf{Студент:}              \\
      Рыбьянов~Антон Валерьевич      \\[7ex]

      \textbf{Научный руководитель:} \\
      Дмитрий Орлов
    \end{tabular}
  \end{flushright}

  \vspace{0pt plus 4fill}
  {Москва, 2023}
\end{center}



\include{annotation.tex}
\include{contents.tex}
\include{introduction.tex}
\include{part1.tex}

\end{document}


