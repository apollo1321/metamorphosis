\RequirePackage[l2tabu, orthodox]{nag}

\documentclass[a4paper, 14pt, oneside, openany]{memoir}

%%==============================
%% Используемые пакеты
%%==============================

\usepackage{fontspec}
\usepackage{polyglossia}
\usepackage{geometry}
\usepackage{hyperref}
\usepackage{microtype}
\usepackage[
  backend=biber,
  language=autobib,
  bibencoding=utf8,
  autolang=other,
  style=gost-numeric,
  doi=false,
  isbn=false,
  defernumbers=true,
]{biblatex}

%%==============================
%% Настройка библиографии
%%==============================

\toggletrue{bbx:gostbibliography}
\renewcommand*{\revsdnamepunct}{\addcomma}

\addbibresource{external.bib}

%%==============================
%% Настройка страниц
%%==============================

\geometry{a4paper, top=2cm, bottom=2cm, left=3cm, right=1.5cm, nofoot, nomarginpar}

% В MS Office используются неправильные межстрочные интервалы. 
% В латехе межстрочный интверал соответствует определению по ГОСТ 2382-23-2004.
\OnehalfSpacing

\pagestyle{plain}
\setlength{\parindent}{1.25cm}

%%==============================
%% Настройка языков
%%==============================

\setmainlanguage[babelshorthands=true]{russian}
\setotherlanguage{english}

\addto\captionsrussian{\renewcommand{\contentsname}{СОДЕРЖАНИЕ}}

%%==============================
%% Настройка шрифта
%%==============================

\setmainfont{Times New Roman}
\newfontfamily\cyrillicfonttt{Times New Roman}

%%==============================
%% Настройка гиперссылок
%%==============================

\hypersetup{
  linktocpage=true,
  colorlinks,
  linkcolor=[rgb]{0.9,0,0},
  pdflang={ru},
}

%%==============================
%% Настройка оглавления
%%==============================

\renewcommand{\cftchapterdotsep}{\cftdotsep}
\renewcommand{\cftchapterpagefont}{\normalfont}
\renewcommand{\cftchapterleader}{\cftdotfill{\cftchapterdotsep}}
\renewcommand{\cftchaptername}{\chaptername\space}
\renewcommand{\cftchapteraftersnum}{.\space}

%%==============================
%% Настройка оформления глав
%%==============================

\newcommand{\basesectionfont}{\fontsize{14pt}{16pt}\selectfont\bfseries}

\makeatletter
\makechapterstyle{thesis}{
  \chapterstyle{default}
  \setlength{\beforechapskip}{0pt}
  \setlength{\midchapskip}{0pt}
  \setlength{\afterchapskip}{\the\dimexpr 14 pt * 2}
  \renewcommand*{\chapnamefont}{\basesectionfont}
  \renewcommand*{\chapnumfont}{\basesectionfont}
  \renewcommand*{\chaptitlefont}{\basesectionfont}
  \renewcommand*{\chapterheadstart}{}

  \renewcommand*{\printchapternonum}{\center}

  \renewcommand*{\printchaptername}{\center \chapnamefont \@chapapp \space \thechapter}
  \renewcommand*{\chapternamenum}{}
  \renewcommand*{\afterchapternum}{.\space}
  \renewcommand*{\printchapternum}{}
}
\makeatother
\chapterstyle{thesis}

%%==============================
%% Настройка оформления параграфов
%%==============================

\setsecheadstyle{\basesectionfont\center}

\begin{document}

\setsecnumformat{\csname the#1\endcsname\quad}

\thispagestyle{empty}
\begin{center}
  Федеральное государственное автономное образовательное учреждение \\
  высшего образования \\
  <<Московский физико-технический институт (национальный исследовательский университет)>>

  \smallskip
  {\small
    Физтех-школа прикладной математики и информатики \\
    Кафедра анализа данных
  }
\end{center}

\bigskip
{
  \small\noindent
  \textbf{Направление подготовки:} 09.04.01 Информатика и вычислительная техника \\
  \textbf{Направленность (профиль) подготовки:} Анализ данных и разработка информационных систем
}

\vspace{0pt plus 8fill}

\begin{center}
  \textbf{
    \MakeUppercase{Построение эффективной распределенной очереди}\\
  }
  (магистерская диссертация)

  \vspace{0pt plus 3fill}

  \begin{flushright}
    \begin{tabular}{l}
      \textbf{Студент:}              \\
      Рыбьянов~Антон Валерьевич      \\[7ex]

      \textbf{Научный руководитель:} \\
      Дмитрий Орлов
    \end{tabular}
  \end{flushright}

  \vspace{0pt plus 4fill}
  {Москва, 2023}
\end{center}



%%================
\chapter*{АННОТАЦИЯ}
%%================



\microtypesetup{protrusion=false}
\tableofcontents*
\microtypesetup{protrusion=true}



\chapter*{ВВЕДЕНИЕ}
\addcontentsline{toc}{chapter}{ВВЕДЕНИЕ}

Lorem ipsum dolor sit amet, consectetur adipiscing elit, sed do eiusmod tempor
incididunt ut labore et dolore magna aliqua. Ut enim ad minim veniam, quis
nostrud exercitation ullamco laboris nisi ut aliquip ex ea commodo consequat.
Duis aute irure dolor in reprehenderit in voluptate velit esse cillum dolore eu

fugiat nulla pariatur. Excepteur sint occaecat cupidatat non proident, sunt in
culpa qui officia deserunt mollit anim id est laborum. Lorem ipsum dolor sit
amet, consectetur adipiscing elit, sed do eiusmod tempor incididunt ut labore

et dolore magna aliqua. Ut enim ad minim veniam, quis nostrud exercitation
ullamco laboris nisi ut aliquip ex ea commodo consequat. Duis aute irure dolor
in reprehenderit in voluptate velit esse cillum dolore eu fugiat nulla
pariatur. Excepteur sint occaecat cupidatat non proident, sunt in culpa qui
officia deserunt mollit anim id est laborum.

Lorem ipsum dolor sit amet, consectetur adipiscing elit, sed do eiusmod tempor
incididunt ut labore et dolore magna aliqua. Ut enim ad minim veniam, quis
nostrud exercitation ullamco laboris nisi ut aliquip ex ea commodo consequat.
Duis aute irure dolor in reprehenderit in voluptate velit esse cillum dolore eu
fugiat nulla pariatur. Excepteur sint occaecat cupidatat non proident, sunt in
culpa qui officia deserunt mollit anim id est laborum.

Lorem ipsum dolor sit amet, consectetur adipiscing elit, sed do eiusmod tempor
incididunt ut labore et dolore magna aliqua. Ut enim ad minim veniam, quis
nostrud exercitation ullamco laboris nisi ut aliquip ex ea commodo consequat.
Duis aute irure dolor in reprehenderit in voluptate velit esse cillum dolore eu
fugiat nulla pariatur. Excepteur sint occaecat cupidatat non proident, sunt in
culpa qui officia deserunt mollit anim id est laborum.



\chapter{Обзор существующих решений}
\section{Тест}

Lorem ipsum dolor sit amet, consectetur adipiscing elit, sed do eiusmod tempor
incididunt ut labore et dolore magna aliqua. Ut enim ad minim veniam, quis
nostrud exercitation ullamco laboris nisi ut aliquip ex ea commodo consequat.
Duis aute irure dolor in reprehenderit in voluptate velit esse cillum dolore eu



\chapter{Распределенные очереди}
\section{Kafka}

Lorem ipsum dolor sit amet, consectetur adipiscing elit, sed do eiusmod tempor
incididunt ut labore et dolore magna aliqua. Ut enim ad minim veniam, quis
nostrud exercitation ullamco laboris nisi ut aliquip ex ea commodo consequat.
Duis aute irure dolor in reprehenderit in voluptate velit esse cillum dolore eu



\chapter{Описание алгоритма}

Lorem ipsum dolor sit amet, consectetur adipiscing elit, sed do eiusmod tempor
incididunt ut labore et dolore magna aliqua. Ut enim ad minim veniam, quis
nostrud exercitation ullamco laboris nisi ut aliquip ex ea commodo consequat.
Duis aute irure dolor in reprehenderit in voluptate velit esse cillum dolore eu



\chapter{Реализация и тестирование}

Lorem ipsum dolor sit amet, consectetur adipiscing elit, sed do eiusmod tempor
incididunt ut labore et dolore magna aliqua. Ut enim ad minim veniam, quis
nostrud exercitation ullamco laboris nisi ut aliquip ex ea commodo consequat.
Duis aute irure dolor in reprehenderit in voluptate velit esse cillum dolore eu



\chapter*{ЗАКЛЮЧЕНИЕ}
\addcontentsline{toc}{chapter}{ЗАКЛЮЧЕНИЕ}

Lorem ipsum dolor sit amet, consectetur adipiscing elit, sed do eiusmod tempor
incididunt ut labore et dolore magna aliqua. Ut enim ad minim veniam, quis
nostrud exercitation ullamco laboris nisi ut aliquip ex ea commodo consequat.
Duis aute irure dolor in reprehenderit in voluptate velit esse cillum dolore eu

fugiat nulla pariatur. Excepteur sint occaecat cupidatat non proident, sunt in
culpa qui officia deserunt mollit anim id est laborum. Lorem ipsum dolor sit
amet, consectetur adipiscing elit, sed do eiusmod tempor incididunt ut labore

et dolore magna aliqua. Ut enim ad minim veniam, quis nostrud exercitation
ullamco laboris nisi ut aliquip ex ea commodo consequat. Duis aute irure dolor
in reprehenderit in voluptate velit esse cillum dolore eu fugiat nulla
pariatur. Excepteur sint occaecat cupidatat non proident, sunt in culpa qui
officia deserunt mollit anim id est laborum.

Lorem ipsum dolor sit amet, consectetur adipiscing elit, sed do eiusmod tempor
incididunt ut labore et dolore magna aliqua. Ut enim ad minim veniam, quis
nostrud exercitation ullamco laboris nisi ut aliquip ex ea commodo consequat.
Duis aute irure dolor in reprehenderit in voluptate velit esse cillum dolore eu
fugiat nulla pariatur. Excepteur sint occaecat cupidatat non proident, sunt in
culpa qui officia deserunt mollit anim id est laborum.

Lorem ipsum dolor sit amet, consectetur adipiscing elit, sed do eiusmod tempor
incididunt ut labore et dolore magna aliqua. Ut enim ad minim veniam, quis
nostrud exercitation ullamco laboris nisi ut aliquip ex ea commodo consequat.
Duis aute irure dolor in reprehenderit in voluptate velit esse cillum dolore eu
fugiat nulla pariatur. Excepteur sint occaecat cupidatat non proident, sunt in
culpa qui officia deserunt mollit anim id est laborum.



\microtypesetup{protrusion=false}
\printbibliography[title={СПИСОК ИСПОЛЬЗОВАННЫХ ИСТОЧНИКОВ}]
\microtypesetup{protrusion=true}




\end{document}


